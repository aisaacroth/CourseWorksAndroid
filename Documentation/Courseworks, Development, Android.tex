\documentclass{article}
\usepackage[utf8]{inputenc}
\usepackage{indentfirst}
\usepackage{soul}

\title{Courseworks App Development Documentation \\ Pre-Alpha Build}
\author{Alexander Roth \\ UNI: air2112}
\date{July 7th, 2014}

\begin{document}
\maketitle

% Section 1: Overview
\section{Overview}
The goal of mobile application is to compliment the features of Columbia's Courseworks web
site and increase accessibility for students. \\
\indent As of \date{\today}, this document outlines the development process and future 
iterations of the Columbia Courseworks app. 
\textbf{Note:} This application is still in development in a \emph{pre-alpha build} and is
currently being built for Android only. Currently looking into development for iOS
devices.

% Section 2: Features
\section{Features}
Again, this application is in the pre-alpha development cycle. Thus, this section is 
subject to heavy change depending on justification in further design and development 
cycles.

% Subsection 2.1: Current Features
\subsection{Current Features}
Features currently implemented within the application include:
\begin{enumerate}
    \item   ``Remember Me!'' functionality
    \item   REST API calls to CAS server for Log In procedures.
    \item   Dynamic Notification center.
\end{enumerate}

% Subsection 2.2: Future & Proposed Features
\subsection{Future \& Proposed Features}
\begin{enumerate}
    \item   Current Semester Homepage
    \item   Course Names instead of Course numbers
    \item   Seamless calendar integration with native mobile calendar applications.
    \item   A viewable roster
    \item   Easy downloads under ``Files \& Resources''
\end{enumerate}

% Section 3: Implementation & Efficiency
\section{Implementation \& Efficiency}

% Subsection 3.1: Workflow
\subsection{Workflow}
The current design of the app works in a three-step process: the preamble, the authenticat
ion, and the main process.
\begin{enumerate}
    \item \textbf{Preamble Process} \\ 
        The preamble process involves the use of the Splash screen. To the user, a splash 
        screen with Columbia's logo will be displayed. On the back-end, the application 
        will be checking if the ``auth.xml'' exists. If this file is present, then a login
        attempt will begin. If this file is not present, then the login activity will be 
        invoked.
    \item \textbf{Authentication Process} \\ 
        After the preamble process, the application moves into authentication. If the auth
        file exists, the system goes to auto-login using the credentials of the last user.
        If this is not the case, the user will be redirected to the login screen, where he
        or she will provide credentials for the system to use. \\ Once credentials are 
        provided, the application launches a series of RESTful calls to CAS servers, which
        then authenticates the user. Once the user has been authenticated, the application
        moves on towards the information gathering process.
        
    \item \textbf{Information Process} \\
        The main process of the application. Once the user is authenticated, we implement
        some method that will recreate the user's identity. After the reconstruction of
        the user's identity, we send out a number of GET requests to the Courseworks API,
        which will allow the application to parse the information in as JSON or XML.
\end{enumerate}

% Subsection 3.2: Allocation of Resources
\subsection{Resource Allocation}
Since this application is native to the mobile device, there will be very little 
overhead, except for the HTTP requests and responses when communicating with the 
servers. As of \date{\today}, there are only two requests being sent to the Courseworks 
servers: The POST and GET requests from the Authentication process.

% Section 4: Class Design
\section{Class Design}

% Subsection 4.1: AnnoucnementView class
\subsection{AnnnouncementView}
A fragment activity that acts within the Main activity. It is the first activity that is
launched after authentication. It will take the parsed JSON and display it dynamically
to the user in a list format.

% Subsection 4.2: AuthPreferences class
\subsection{AuthPreferences}
A helper class for the Login activity. The class contains a number of methods that either
write to or read from the SharedPreference file ``auth.xml''.

% Subsection 4.3: CalendarView class
\subsection{CalendarView}
Currently a work in progress: the CalendarView will act as a Calendar showing upcoming
events for a user within a given time frame. Possibly considering updating that for just
one week display. Also, implementing a date picker feature to add events.

% Subsection 4.5: CourseView class
\subsection{CourseView}
Currently a work in progress: the CourseView will show all the current classes in a user's
current semester. Need to implement a method that launches an activity that will display
information relating to that class from that view.

% Subsection 4.5: InformationRequester class
\subsection{InformationRequester}
The backend of the application. Handles all the GET and POST calls to the Courseworks API.

% Subsection 4.6: Main class
\subsection{Main}
The main hub of the application. It holds the fragments that act as different acts for
each section. 

% Subsection 4.7: Login class
\subsection{Login}
This activity controls the Login screen. View the source and javadocs for specific 
functionality for each method.


% Subsection 4.8: RestGrant class
\subsection{RestGrant}
An alternative to both OAuth implementations mentioned beforehand. Implements CAS RESTful 
API to make a series of calls to the CAS Authentication servers.

% Subsection 4.9: Splash class
\subsection{Splash} 
The Splash activity acts as a buffer between the Login activity (on first access of the 
application) or the Main activity on the application. 


% Section 5: Risks
\section{Risks}
\subsection{Security}
Since the application is native, it is more conducive to store the user's credentials on 
the mobile device; thus allowing the application to hold the user's credentials in a file 
system. However, this ease opens up a large security risk. The file holding the user's 
sensitive information (e.g. uni and password) is not secure at the present moment. There 
are three workarounds to this issue:
 \begin{enumerate}
    \item   We encrypt the SharedPreference file holding the user's information.
    \item   We remove the ``Remember Me!'' feature.
    \item   We move to web authentication through OAuth's \emph{Authorization Code}     
            protocol.
    \item   We move to web authentication through OAuth's \emph{Implicit Grant} protocol.
\end{enumerate}
As of \today, I have implemented an encryption method using Advanced Encryption Standard (AES). However, the SharedPreference file is not guaranteed completely secure; I just increased the difficulty for someone to access the information.

% Section 6: ToDo
\section{ToDo}
\begin{enumerate}
    \item   Test Authentication Process.
    \item   \st{Secure private information of user.}
    \item   Update feature list based on student response.
    \item   Move to alpha build once the Courseworks servers are prepared.
    \item   Implement methods to the Courseworks API.
\end{enumerate}

\end{document}